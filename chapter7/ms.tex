\justifying

%%----------------------------------------------------
\section{Introduction} \label{sec:intro}
%%----------------------------------------------------

The Solar Ultraviolet Imaging Telescope onboard {\it Aditya-L1} \citep[{\it Aditya-L1}/SUIT,][]{article,ghosh16,adityal1,suit_main} provides a targeted probe into the Chromosphere and Transition region. It provides continuous full-disk and Region of Interest (RoI) coverage of the Sun in eleven pass bands. The details of these bands are provided in chapter~\ref{c:chap3} Tab.~\ref{tab:science_filters}. The eight narrow bands provide coverage across the \ion{Mg}{2} k and h lines, \ion{Ca}{2} h line, the CN band, red and blue wing of the \ion{Mg}{2} window and parts of the NUV continuum. As alluded earlier, this provides unprecedented coverage of the Chromospheric and Transition region structure of solar flares.

The {\it Aditya-L1} was launched on September 2nd, 2023. The first light observations were made on December 5th, 2023 while the payload was still in cruise phase. For the majority of the cruise phase the payload only operated in the 2k synoptic mode, i.e. it only captured continuous (2k $\times$ 2k) observations in NB4 (\ion{Mg}{2} h line) with one minute cadence. The L1 insertion was carried out on January 6th, 2024. Following that the payload also started taking (4k $\times$ 4k) observations in all eleven science filters. Various components of the flare detection algorithm were progressively turned on. In this chapter we discuss some of the initial flare observations made by SUIT. The first flare triggered by the flare detection algorithm was a X6.3 flare from north-west corner of the disk, which was also observed by several other instruments {\it e.g.} {\it SDO}/AIA, {\it IRIS}, {\it SO}/STIX.  