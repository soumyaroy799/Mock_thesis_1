This thesis addressed several questions regarding solar flares and their effects on the local plasma environment. We first estimate thermal energies in two solar flares. We use DEM analysis with AIA, XRT and SUVI data to estimate the thermal energy of these flares as a function of time. We describe a method to use stereoscopic observations from AIA and {\it STEREO-A} data to accurately determine the flare arcade volume as a function of time. We also describe multiple initial preparatory analyses we carried out for the design and calibration for {\suit}. We first describe the analysis we carried out with the simulated spacecraft jitter data provided by the ISRO URSC team to quantify the RMS jitter as a function of exposure time. Following this, we describe the throughput model of {\suit} and describe how this method was used to choose the science filters to be mounted on {\suit}. Subsequently, we describe the on-orbit stellar calibration plan of {\suit} using Sirius-A as a calibration source, which is yet to be carried out. We then describe the forward model pipeline we set up to create mock {\suit} observations to characterize the imaging performance and the effects of the instrument PSF on the imaging. Following this, we use IRIS \ion{Mg}{2} data to investigate the effects of Solar flares on the local plasma environment. We describe how similar observations can be used to regularly monitor the similar effects from {\suit} NB3 and NB4 observations. Finally, we describe two of the first flares observed by {\suit}. The main results of this thesis can be summarised as follows:

%%%%%%%%%%%%%%%%%%%%%%%%%%%%
\begin{enumerate}
    \item The thermal energy estimation for solar flares can be significantly affected by the volume estimates of the flaring plasma in the impulsive phase. For the November 29, 2020, limb event, we also demonstrated directly from the calculated thermal energy that the fan was cooling down slower compared to the loops, indicating a different heating mechanism. This has already been alluded to in various existing studies \citep[e.g.,][]{xie23,reeves19,longcope11}.

    \item The measured PSF across various channels show $\sim~2\arcsec$ radius of 80\% FWHM for the PSF. This illustrates that the imaging from {\suit} would be able to resolve features $\sim~2\arcsec$ apart. 
    
    \item The \ion{Mg}{2} k \& h line intensity ratios can be used to probe the effects of solar flares on the local plasma environment. We believe the correlated change in the line intensity ratio with the SXR light curve is a direct effect of localised heating and an increase in the local optical depth due to the down-flow from the flaring region.

    \item The off-limb plasma blob observed in the December 31st, 2023 event is chromospheric cold material ejected into the atmosphere via an eruption. The blob is then seen accelerated. The acceleration of the plasma blob coincides with the development of a kink on the overlying loop. The acceleration occurs due to ongoing bursty reconnection from the same active region, which erupts into a second flare.

    \item The on-disk flare observed on February 22nd 2024, exhibits penumbral bright kernels in the blue and red wings of \ion{Mg}{2} along with white light. Similar observations were made for a separate event by \cite{kowalski19}. These bright kernels originate due to several metallic lines going into emission in both wings. The red wing enhancement occurs predominantly due to several \ion{Fe}{2} lines. The local plasma at photospheric heights is photoionized by the $\ge$ 7.12 keV SXR, giving rise to the several \ion{Fe}{2} lines in the red wing of \ion{Mg}{2}. This also gives rise to several atomic transitions in the Fe atoms, giving rise to Fe line fluorescence which is observed via the XSM data. Our result proposes a separate possible mechanism that can give rise to penumbral bright kernels instead of previously suggested direct heating by the electron beam (refer to \cite{kowalski19} for further details). 
    
\end{enumerate}  
%%%%%%%%%%%%%%%%%%%%%%%%%%%%

These results open up several pathways to explore and pressing questions to answer:

\noindent {\bf Why was changes in \ion{Mg}{2} k to h line intensity ratio not observed in the X-class flare?} We require further analysis with multiple flares of same classes to investigate why the changes in the line intensity ratio was not visible in the X-class flare. There are two distinct possibilities:

%%%%%%%%%%%
    \begin{itemize}[label=\ding{226}]
        \item It is possible that the change was not observed for the X class flare, simply due to the cadence limitations of IRIS raster scan, along with which regions of the flare arcade is being scanned by IRIS because it is well known that the \ion{Mg}{2} profiles vary significantly in shape and spatially within the flaring region \citep{panos18,dalda23}.
        \item The other, more interesting possibility is that in some of the events depending on the "impulsiveness" of the event, there might be different degrees of ionisation at play at the chromosphere. This also might indicate a different energy release mechanism during the preflare and impulsive phase.
        %\item One of the crucial tools in this regard would be finding correlations between {\it IRIS} and {\suit} observations. {\it IRIS} has been taking regular planned observations for {\suit} in QS and AR. One of the studies we will conduct immediately, is correlate {\it IRIS} intensities with {\suit} counts using these observations. As highlighted in \cite{roy24}, the {\it IRIS} intensities can be used to probe the local optical depth in the Mg lines. The correlation would immediately allow us to comment on the full-disk \ion{Mg}{2} k to h line intensity ratio continuously across various features using {\suit} observations.
    \end{itemize}
    %%%%%%%%%%%

\noindent {\bf Continuous full-disk observation of opacity using {\suit} NB3 and NB4 observations,} with the observations highlighted in \cite{roy24}. This provides us with continuous full-disk opacity diagnostic throughout the day, across various solar features and eruptive events, and study any possible correlated photospheric and/or coronal signatures.

%\noindent {\bf How important are high resolution imaging from different vantages?} Our results emphasizes the critical role of accurate volume estimation in determining the thermal energy of solar flares. Discrepancies in volume estimation, particularly during the impulsive phase and thermal peak, may lead to overestimation of thermal energy because rapid changes in plasma volume occur during these stages. In contrast, during the decay phase, the volume stabilizes, reducing errors in energy calculations. A key challenge previously discussed in \S\ref{sol_flr_energ} is the discrepancy in flare energetics: while larger flares (like M-class or above) hold sufficient energy in non-thermal particles to power thermal components, smaller flares often show a deficit, suggesting the necessity of a secondary energy source. The central question raised is the reliability of thermal energy estimates, especially if they are consistently overestimated during the impulsive and peak phases does that have a role to play in the discrepancy observed for the smaller flares? To address this, rigorous statistical studies across different flare classes with accurate volume measurements are needed to understand how volume estimation impacts thermal energy calculations, which in turn requires high resolution imaging of the flaring plasma from various vantages, to harness the stereoscopic method discussed in chapter \ref{c:chap3}. While our method is described for EUV observations, a similar method in SXR has been described by \cite{ryan24} using {\it Hinode}/XRT and {\it SO}/STIX observations from different vantages to study the 3D evolution of the thermal loop-top source for a M3.9 flare. Currently this is only feasible in EUV with {\it STEREO-A}/EUI and {\it SO}/EUI.

\noindent
{\bf How important are high resolution imaging from different vantages?} Our results emphasize the critical role of accurate volume estimation in determining the thermal energy of solar flares. Discrepancies in volume estimation, particularly during the impulsive phase and thermal peak, may lead to overestimation of thermal energy because rapid changes in plasma volume occur during these stages. In contrast, during the decay phase, the volume stabilizes. A key challenge previously discussed in \S\ref{sol_flr_energ} is the discrepancy in flare energetics: while larger flares (like M-class or above) hold sufficient energy in non-thermal particles to power thermal components, smaller flares often show a deficit, suggesting the necessity of a secondary energy source. The central question raised is the reliability of thermal energy estimates, especially if they are consistently overestimated during the impulsive and peak phases—does this contribute to the discrepancy observed in smaller flares? To address this, rigorous statistical studies across different flare classes with accurate volume measurements are needed to understand how volume estimation impacts thermal energy calculations. This, in turn, requires high-resolution imaging of the flaring plasma from various vantages, to harness the stereoscopic method discussed in Chapter~\ref{c:chap3}. While our method is described for EUV observations, a similar method in SXR has been outlined by \cite{ryan24}, using \textit{Hinode}/XRT and \textit{SO}/STIX observations from different vantages to study the 3D evolution of the thermal loop-top source for an M3.9 flare. Currently, this is only feasible in EUV with \textit{STEREO-A}/EUI and \textit{SO}/EUI.


\noindent
{\bf Importance of current {\suit} flare observations and what are the future prospects?} {\suit} provides first full-disk solar imaging in all of the concerned pass bands. There are previous flare observations in some of the imaging channels {\it e.g.} NB3, NB4 and NB5 from {\it IRIS} and NB8 from {\it Hinode}/SOT, only in smaller spatial windows. One of the key disadvantage of smaller spatial windows is the possibility of missing events due to different pointing. {\suit}, equipped with its flare detection algorithm, has already observed several flares including various limb events. This provides us our first observation of off-limb flares in all these lines. A larger statistical study with such observations combined with EUV, SXR, HXR, H$\alpha$, and WL observations can provide us with more complete picture of how flare energy is deposited across various layers of the solar atmosphere.