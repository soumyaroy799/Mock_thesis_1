\justifying
%XXXXXThis thesis addressed several questions regarding solar flares and their effects on the local plasma environment. We first estimate thermal energies in two solar flares. We use DEM analysis with AIA, XRT and SUVI data to estimate the thermal energy of these flares as a function of time. We describe a method to use stereoscopic observations from AIA and {\it STEREO-A} data to accurately determine the flare arcade volume as a function of time. We also describe multiple initial preparatory analyses we carried out for the design and calibration for {\suit}. We first describe the analysis we carried out with the simulated spacecraft jitter data provided by the ISRO URSC team to quantify the RMS jitter as a function of exposure time. Following this, we describe the throughput model of {\suit} and describe how this method was used to choose the science filters to be mounted on {\suit}. Subsequently, we describe the on-orbit stellar calibration plan of {\suit} using Sirius-A as a calibration source, which is yet to be carried out. We then describe the forward model pipeline we set up to create mock {\suit} observations to characterize the imaging performance and the effects of the instrument PSF on the imaging. Following this, we use IRIS \ion{Mg}{2} data to investigate the effects of Solar flares on the local plasma environment. We describe how similar observations can be used to regularly monitor the similar effects from {\suit} NB3 and NB4 observations. Finally, we describe two of the first flares observed by {\suit}. The main results of this thesis can be summarised as follows:

%\textbf{For the above paragraph -- just compress the three paragraph in the synopsis into one. Then say that we summarise the finding in this thesis below.}

%The below items has to be like abstract for each chapter... 

Solar flares are the most powerful magnetic events in the solar system, characterized by sudden, localized brightening in the solar atmosphere. %These eruptions are triggered by magnetic reconnection, releasing vast amounts of energy in the form of radiation and energetic particles. Solar flares often produce large-scale disturbances, including filament or prominence eruptions, coronal mass ejections (CMEs), and wave phenomena such as Moreton waves and EUV waves. 
The energy released during a flare can affect space weather and disrupt systems such as satellite communications, GPS, power grids, etc., including serious risks to astronauts and spacecraft in orbit. Studying the underlying physics of solar flares helps us better predict these events and mitigate their impacts, which is critical for the safety of both space-based infrastructure and terrestrial technology. %Furthermore, solar flares offer a unique opportunity to study plasma physics processes, such as magnetic reconnection, which are central to solar physics, astrophysics, and even fusion research. 
The extensive observations of flares, especially in soft X-ray, hard X-ray, and EUV wavelengths, we have developed a tremendous understanding of the physical processes at work in flares. However, numerous important questions about the origin and evolution of thermal and non-thermal energies in solar flares, their spectral distribution, and the precise mechanisms behind flare initiation remain. 

The primary aim of this thesis is to perform a multi-wavelength study of solar flares using existing observations and also contribute towards the development of the new observing facility, the Solar Ultraviolet Imaging Telescope (SUIT) onboard Aditya-L1 that provides crucial data for flare studies. The thesis is structured in two parts. In the first part, we address some of these questions by analyzing the temporal evolution of thermal and non-thermal energies in solar flares using EUV and X-ray data from multiple vantage points. In the second part, we conduct preparatory studies for SUIT that include modeling the instrument's performance, assessing spacecraft jitter profiles, establishing calibration methods using stellar observations, and forward modeling of mock SUIT observations. Below, we provide a detailed summary of the work and results that were obtained. %  In preparation for SUIT operations, we also analyze spectral observations in the near-ultraviolet range and initial flare observations, setting a foundation for future flare studies with SUIT. Finally, we highlight and discuss some of the first flare observations carried out with {\suit} and its implications on our understanding of solar flares. The main results of this thesis can be summarized as follows:

%%%%%%%%%%%%%%%%%%%%%%%%%%%%
\begin{enumerate}

    \item The partition between the thermal and the non-thermal energies in solar flares informs us about the relevant physical process taking place during the evolution of the flare. As discussed in \S\ref{sol_flr_energ}, several statistical studies of solar flare energetics \cite[e.g.,][]{warmuth16a, warmuth16b, stosire07, emslie12, inglis14, ash17} have reported on the partition between the thermal and non-thermal energies of the flares. It can be inferred that for bigger flares (e.g., M and X class), there is enough non-thermal energy in electrons to power the thermal radiation. In comparison, there seems to be a deficit in the non-thermal component for the smaller flares. This prompted speculation about a third energy source that could account for the deficit of energy needed to power the thermal component in smaller flares. One of the key similarities of all of these studies was the use of peak thermal energy as the representative of the total thermal energy of the flares. While that is a fair representation of the overall thermal energy budget of an event, it can miss several intricacies throughout the evolution of the flare. In this thesis, we study two flares, one on disk and the other off-limb. We use AIA and XRT observations to compute DEMs and infer the thermal properties of the flaring plasma. We use observations from AIA and EUVI, from a different vantage,  to triangulate the LoS through the flaring plasma. We find that the thermal energy estimation for solar flares can be significantly affected by the volume estimates of the flaring plasma in the impulsive phase. For the November 29, 2020, limb event, we also demonstrated directly from the calculated thermal energy that the cooling pattern of fans is different than the post-flare loops, thereby indicating a possibly different heating mechanism. A hint of such observations has already been alluded to in various existing studies \citep[e.g.,][]{xie23,reeves19,longcope11}. 

    \item Estimating the imaging performance of {\suit} was one of the priorities to characterize and quantify the possible science cases. To this end, we designed a pipeline to create mock {\suit} observations using simulated MURaM cubes. We incorporated lab-measured PSF in the pipeline to create realistic mock observations.  The measured PSF across various channels show $\sim~2\arcsec$ radius of 80\% FWHM for the PSF. This illustrates that the imaging from {\suit} would be able to resolve features $\sim~2\arcsec$ apart with reliable photometric accuracy. 

    \item The intensity ratios of the \ion{Mg}{2}~k~and~h lines can be used to probe the characteristics of the plasma in the {\bf lower} solar atmosphere {\bf in chromosphere and transition region heights}. To obtain the plasma characteristics in flaring regions, we study the variation of \ion{Mg}{2}~k~and~h intensity ratio for three areas belonging to X-class, M-class, and C-class throughout their evolution. For this purpose, we used existing IRIS observations combined with those from AIA, HMI and GOES X-ray light curves. We co-aligned IRIS 2796~{\AA} observations with those from HMI and obtained artificially rastered magnetic field maps corresponding to IRIS maps. This gives us the measure of the photospheric magnetic field for every pixel of the IRIS rasters. We use a double Gaussian function with a symmetric background about the line core to fit the \ion{Mg}{2} k and h lines. In scenarios where the Mg triplet lines are in emission, they are fitted with separate Gaussian functions. We find that the intensity ratios show significant changes during flares, i.e., it peaks minutes before the GOES SXR peak and falls even below the pre-flare level during the peak and decline phase of the flare. A comparison with photospheric magnetic flux density suggests that the change in ratio is independent of flux density. Given that the \ion{Mg}{2} k to h line intensity ratio is representative of the opacity of the local plasma environment, these results are important in the light of heating and cooling of localized plasma and provide further constraint on the understanding of flare physics.

    \item The several flares observed via {\suit} give us a new perspective on observing various eruptions and how they interact with the local plasma environment.  Here, we report the observation of the first flare X6.3 flare SOL2024-02-22T22:08:00 that was localized and observed with the eight narrow band (NB) filters of SUIT. We have also used the co-spatiotemporal observations from SDO/AIA, SolO/STIX and Global Oscillation Network Group (GONG) H$\alpha$ and GOES. We align and co-register SUIT observations with those from AIA 1600, 1700~{\AA} and GONG H$\alpha$ observations and construct light curves from the same regions on them. We compare these light curves with the full-disk integrated GOES SXR and STIX HXR (25{--}50~keV) light curves. We find that all of the SUIT NB1, NB3, NB4, NB8 filters peak simultaneously with HXR and 1600, 1700~{\AA}. In contrast, the NB3 and NB5 lines peak $\sim$~3 minutes later than the HXR peak. The flare peaks in NB6 and NB7 $\sim$~3 mintes after the GOES soft X-ray peak. To the best of our knowledge, this is the first observation of flare in these wavelengths (except in NB3, NB4 and NB5). Moreover, for the first time, we show the presence of bright kernel NB2. These results demonstrate the capabilities of SUIT observations in flare studies.  
    
\end{enumerate}  
%%%%%%%%%%%%%%%%%%%%%%%%%%%%

The results obtained in this thesis help us understand various puzzles of solar flares and provide a number of constraints on the modeling. Additionally, they open up several pathways to further explore the physics of flares. The observations from the SUIT instrument open up a new window for studying solar flares. Below we describe a few projects that naturally arise based on this study. 

%%%%%%%%%%%
\begin{itemize}[label=\ding{226}]

\item Our results in Chap.~\ref{c:chap3} emphasize the critical role of accurate volume estimation in determining the thermal energy of solar flares. Discrepancies in volume estimation, particularly during the impulsive phase and thermal peak, may lead to overestimation of thermal energy because rapid changes in plasma volume occur during these stages. Therefore, we plan to automate the triangulation method to determine the LoS from earth vantage using observations from various instruments and conduct rigorous statistical studies across different flare classes to understand how volume estimation impacts thermal energy calculations. 

%This, in turn, requires high-resolution imaging of the flaring plasma from various vantages to harness the stereoscopic method discussed in chap.~\ref{c:chap3}. While our method is described for EUV observations, a similar method in SXR has been outlined by \cite{ryan24}, using \textit{Hinode}/XRT and \textit{SO}/STIX observations from different vantages to study the 3D evolution of the thermal loop-top source for an M3.9 flare. Currently, this is only feasible in EUV with \textit{STEREO-A}/EUI and \textit{SO}/EUI.
         %\end{enumerate}
        
\item As alluded to in Chap.~\ref{c:chap6}, we did not observe the correlated change in line intensity rations for the X-class flare. It is possible that the change was not observed for the X class flare, simply due to the cadence limitations of IRIS raster scan, along with which regions of the flare arcade are being scanned by IRIS because it is well known that the \ion{Mg}{2} profiles vary significantly in shape and spatially within the flaring region \citep{panos18,dalda23}. The other and more interesting possibility is that in some of the events, depending on their "impulsiveness", there might be different degrees of ionization at play in the chromosphere. This may also indicate a different energy release mechanism during the pre-flare and impulsive phases of the flare.

\item {\suit} provides the first full-disk solar imaging in 11 filters in the wavelength range of 200{--}400~nm. There are previous flare observations in some of the imaging channels {\it, e.g.} NB3, NB4 and NB5 from {\it IRIS} and NB8 from {\it Hinode}/SOT, only in smaller spatial windows. %Therefore, SUIT observations open up a new window for flare studies. One of the key disadvantages of smaller spatial windows is the possibility of missing events due to different pointing. {\suit}, equipped with its flare detection algorithm, has already observed several flares, including various limb events. This provides us our first observation of off-limb flares in all these lines. A larger statistical study with such observations combined with EUV, SXR, HXR, H$\alpha$, and WL observations can provide a more complete picture of how flare energy is deposited across various layers of the solar atmosphere. We specifically want to explore the observation of the probable flare lines observed in the blue wing of \ion{Mg}{2} discussed in chap.~\ref{c:chap7}. We plan to use flare simulations, together with observations from IRIS and XSM when available to comment on the spectral nature and origin of the penumbral bright flare kernels.
        
    \end{itemize}
%%%%%%%%%%%


