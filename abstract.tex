\justifying

\abstract{

Solar flares are the largest magnetic eruptions in the Solar system. The enigmatic magnetic eruptions have been studied for over a century. However, their complete spectral energy distribution is not yet derived. One of the main missing components is flare energetics in the photosphere and chromosphere. While there are many measurements of flare energetics in the extreme-ultraviolet (EUV), soft and hard X-rays, the measurements in other wavelength bands, especially in the near ultraviolet, are sparse. The Solar Ultraviolet Imaging Telescope ({\it SUIT}) on board Aditya-L1 by the Indian Space Research Organisation (ISRO) aims to fill the gap by observing the flares in the wavelength band 200 {--} 400 nm. Moreover, using the EUV and X-ray observations, there has been great progress in flare physics. However, there is a lack in our understanding of the evolution of thermal and non-thermal energy during the course of the evolution of flares and the partition between the thermal and non-thermal energy, including the conversion mechanism at play. This thesis is in two parts. In one part, we perform the initial preparatory analysis carried out for the {\it SUIT} instrument, including the throughput model, jitter estimation and its effect on imaging, the on-ground spectral and photometric validation, forward modelling of {\it SUIT} observations using the data obtained from realistic MHD simulations from MURaM, and the stellar calibration plan. We also carry out analysis with existing IRIS data in anticipation of SUIT, to comment on the effect of solar flares in the local environment. In the second part, we use the existing EUV and X-ray observations to study the thermal and non-thermal energy in solar flares and demonstrate the importance of volume estimation on the total thermal energy budget and, thereby, the efficiency of the conversion from non-thermal to thermal energy. We propose a triangulation method using existing tools and observations from different vantage points to estimate the volume of emitting plasma in flares as a function of time. Finally, we study two flares which are observed by SUIT in NUV combined with EUV, NUV, Soft and Hard X-ray spectroscopy and for the first time we discuss the flare observations and initial results in this broad wavelength coverage.

}
