\justifying

\abstract{
Solar flares are the largest magnetic eruptions in the Solar system. The enigmatic magnetic eruptions have been studied for over a century. The effects of flare in the lower solar Atmosphere {\it e.g.} Chromosphere and Photosphere are also very ill-understood. There is also a lack of full-disk resolved observation in continuum channels in the NUV regime. In addition to this, instead of the availability of plethora of solar observations, we still hardly understand various aspects of the flares. Various statistical studies exhibit our lack of understanding between the partition of thermal and non-thermal energy, the conversion mechanism. The {\suit} on-board Aditya-L1 by the Indian Space Research Organisation (ISRO), aims to fill the gap of full disk observation in the NUV regime, specifically in 200 {--} 400 nm.

In this thesis we first describe the initial preparatory analysis carried out for {\suit}, including the throughput model, jitter estimation and its effect on imaging, the on ground spectral and photometric validation and the stellar calibration plan. We also use EUV, NUV, Soft and Hard X-ray spectroscopy and Full-disk integrated imaging to understand: (\romannum{1}) The effect of Solar flares on the surrounding plasma environment in Chromosphere, (\romannum{2}) The effect of volume estimation on the thermal energy budget of the Solar flare and the efficiency of the conversion from non-thermal to thermal energy. We propose a triangulation method using existing tools and observations from different vantages to estimate the volume of the arcade as a function of time. We also report the first flare observations from {\suit} and its implications on our understanding of flares moving forward.

}
