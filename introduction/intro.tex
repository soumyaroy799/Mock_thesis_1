

The Sun is our nearest astrophysical object, which serves as a natural laboratory for various branches of Physics - atomic physics, spectroscopy, turbulence, magnetohydrodynamics, stellar evolution, dynamo and magnetism and plays a key role in space and terrestrial weather. The Sun is also the primary source of light and energy on earth. So it is only natural that we try to understand the different physical processes involved in the energectics of the Sun, which indirecrtly affects life on Earth. In this chapter I briefly introudce the Sun and various layers of its atmosphere. At the end we provide a motivation for the thesis.

The Sun is a G2V star, with surface luminosity $3.86~\times~10^{26}$ W and an effective temperature of $\sim 5780~K$. The Sun is mostky made out of Hydrogen(92.1\%) and Helium(7.8\%) along with negligible quantities of heavier elements C, N, O, Mg, Si, Ne and Fe. The study of the Sun can be mainly divided into three components: the Solar interior, the Solar atmosphere and the Heliosphere.

%%%%%%%%%%%%%%%%%%%%%%%%%%%%%%%%%%%%%%%%%%%%%%%%%%%%%%%
\section{The solar interiror}\label{solar_int}
%%%%%%%%%%%%%%%%%%%%%%%%%%%%%%%%%%%%%%%%%%%%%%%%%%%%%%%

In the innermost zone of the star the Sun has a temperature of about 15 MK and a density of $1.6~\times~10^{5}~km.m^{-3}$ that fuses Hydrogen into Helium mainly by p-p cycle and to some extent by CNO cycle \sr{Ref here}. 

%% ############# %%
\begin{figure}[h!]
    \centering
    \includegraphics[width = 0.8\linewidth]{Figures/solar_int.png}
    \caption{A schematic depiction of the different layers of the Sun. The figure also shows soe of the most prominent magnetic structures on the Sun e.g sunspot, prominence, coronal hole. Credit : NASA/Goddard. \sr{Better pic here.}}
    \label{fig_solar_int}
\end{figure}
%% ############# %%

Fig.~\ref{fig_solar_int} shows a schematic diagram of various layers of the Sun. Beyond the core we have the optically thick radiative zone, where the Hydrogen and Helium are completely ionized \sr{Ref. here}. The high energy $\gamma$-ray photons generated in the core, have very small mean free path in this region, as they collide neumerous times eventually thermalizing becoming visible photon. Beyond the radiative zone the degree of ionization of Hydrogen and Helium changes drastically giving rise to a sharp temperature gardient. This is the convective zone where the energy from the inner layers of the Sun is transported adiabetically to the surface of the Sun. Inbetween the radiative and the convective zone, there is a thin layer, roughly at $\sim~0.7~R_{\odot}$ known as the tacholine, which is believed to play a major role in generating Sun's magnetic field via the Dynamo process \sr{Ref. here}. 

The core and the radiative zone of the Sun rotates as a solid body, whereas the convective zone rotates differentially. The rotation period varies from $\sim~25~days$ to $\sim~35~days$ from the equator to the pole.

%%%%%%%%%%%%%%%%%%%%%%%%%%%%%%%%%%%%%%%%%%%%%%%%%%%%%%%
\subsection{The solar atmosphere}\label{solar_atmos}
%%%%%%%%%%%%%%%%%%%%%%%%%%%%%%%%%%%%%%%%%%%%%%%%%%%%%%%

The layers outside the convection zone together, form the solar atmpshere - the Photosphere, which is also known as the Sun's surface, the Chromosphere, the transition region and the Corona. While we can define geometrical height or depths to define these layers, it is much more useful to define them using the local optical depths of various spectral lines, which are directly used to observe the layers. This also gives us a idea about which spectral features originate mainly from which portions of the solar atmosphere. Fig.~\ref{fig_solar_atm} shows the variation of temperature and number density accross various layers of the Solar atmosphere \sr{Ref. here}.

%% ############# %%
\begin{figure}[h!]
    \centering
    \includegraphics[width = 0.8\linewidth]{Figures/solar_atm.png}
    \caption{The variation of electron temperature (solid line) and electron density (Ne, dashed line) and density of neutral hydrogen atoms (NH, dotted- dashed line) in the solar atmosphere as derived based on the 1-D model calcula- tions by \sr{Ref. here}.}
    \label{fig_solar_atm}
\end{figure}
%% ############# %%

%%%%%%%%%%%%%%%%%%%%%%%%%%%%%%%%%%%%%%%%%%%%%%%%%%%%%%%
\subsubsection{The Photosphere}\label{photosphere}
%%%%%%%%%%%%%%%%%%%%%%%%%%%%%%%%%%%%%%%%%%%%%%%%%%%%%%%

The Photosphere is defined to be the layer where $\tau_{\lambda~\sim~5000}~=~1$. This is in the green part of the visible spectrum and the Sun is opaque in visible beyond this layer, hence it is called to be the surface of the Sun. This layer is 400-600 km deep and has an effective temperature of 5780 K. The magnetic filed lines arising from the tacholine penetrate the Photosphere and creates a `carpet' over the whole region \sr{(Priest 2014) ref here}. As mentioned earlier, this is the Solar surface known as the `Quite Sun'(QS from hereon). This region exhibits an average magnetic flux density of \textbf{10 -- 50~G}. The QS surface is covered with cells of roughly four size- granule, mesogranules, supergranules and giant cells. Some regions exhibit much sronger magnetic flux density often asociated with highly twisted magnetic field structures. These magnetic features are manifested as sunspots, spcules, bright points etc. 

As the density changes drastically at the Photosphere compared to the convective zone, the thermalized photons from Sun's core start freestreaming again, as the mean free path also increases drastically and as a result perturbs the thermal equlibrium (TE from hereon). So, we have to define the thermodynamic quantites of the Photosphere in Local Thermal Equlibrium (LTE from hereon). As we move away from the Photosphere, because the density keeps on decreasing steadily, the LTE conditiobs also start deviating similarly \sr{Ref here}.

%%%%%%%%%%%%%%%%%%%%%%%%%%%%%%%%%%%%%%%%%%%%%%%%%%%%%%%
\subsubsection{The Chromosphere}\label{chromosphere}
%%%%%%%%%%%%%%%%%%%%%%%%%%%%%%%%%%%%%%%%%%%%%%%%%%%%%%%

The Chromosphere is a highly non-uniform dynamic layer with a thickness of 1500~--~3000~km, with increasing temperature (upto $10^{4}$ K) and decreasing numbr density. As seen in the fig. \ref{fig_solar_atm}, the temperature of the Chromosphere saturates before the dramatic rise in the transition region. This saturation is attributed to a steady deposition of accoustic energy by creation of shock waves \sr{Ref. here}. The Chromosphere also exhibits a sharp gradient in the plasma $\beta$ factor, non-LTE conditions, dominance of wave motions \sr{Ref. here}. 

%%%%%%%%%%%%%%%%%%%%%%%%%%%%%%%%%%%%%%%%%%%%%%%%%%%%%%%%%%%%%%%%%%%
\subsubsection{The Transition Region}\label{transition-region}
%%%%%%%%%%%%%%%%%%%%%%%%%%%%%%%%%%%%%%%%%%%%%%%%%%%%%%%%%%%%%%%%%%%

The transition region above the Chromosphere is a thin, dynamic layer where we witness a dramatic increase in temperature by two orders of magnitude along with a similar decrease in electron density by nearly 5-6 orders of magnitude which is demonstrated in fig. \ref{fig_solar_atm}. In general the this layer is roughly $\sim$ 100 km in thickness, but that can change depending on various dynamic conditions of Chromosphere($\sim~10^{4}~K$) below, or Corona($\sim~10^{6}~K$) on top. The transition region is generally characterized by the steep change in the temperature, pressure gradients, drastic change in local optical depth and competition between gas pressure and magnetic pressure. The transition region also manifests the ARs as various magnetic features such as small scale brightenings, jets, spicules, fibrils etc. 

%%%%%%%%%%%%%%%%%%%%%%%%%%%%%%%%%%%%%%%%%%%%%%%%%%%%%%%%%%%%%%%%%%%
\subsubsection{The Corona}\label{corona}
%%%%%%%%%%%%%%%%%%%%%%%%%%%%%%%%%%%%%%%%%%%%%%%%%%%%%%%%%%%%%%%%%%%

The outermost layer of the Solar atmosphere is the Corona. The Corona is only visible in naked eyes during the total solar eclipse or via Coronagraphs.






%%%%%%%%%%%%%%%%%%%%%%%%%%%%%%%%%%%%%%%%%%%%
\section{Solar Flares}\label{sol_flr}
%%%%%%%%%%%%%%%%%%%%%%%%%%%%%%%%%%%%%%%%%%%%

Solar flares are the most powerful magnetic events in the solar system. They are described as a sudden increase in brightness in localized areas on Sun. Within tens of minutes, they can release over $10^{32}$ erg of energy, which is emitted across the entire electromagnetic spectrum from radio to gamma rays. They can also launch high energetic particles into the interplanetary medium. Most of the flares occur in magnetic active regions, and the amount of flare energy released is comparable to the free energy stored in the magnetic system. The term "flare" is generally used explicitly for the entire magnetically-driven event's electromagnetic radiation, as it is the most significant fraction of the total energy liberated. The total energy released varies from event to event. It is also known that larger events occur much less frequently than smaller events.

%% ##################################################### %%
\subsection{Brief history of flare observation}\label{sol_flr_1}
%% ##################################################### %%

In September 1, 1859, R.C. Carrington and R. Hodgson observed the first flare in the continuum of white light\citep{carrington1859,hodgson1859}. The localized brightenings on the Sun have remained an enigma ever since. We have been observing the local flaring events acrsoss all wavelenths from both ground and space based observatories.  Shortly after the observation by Carrington and Hodgson, people started studying the Sun extensively in the H$\alpha$ line which is formed in the Chromosphere, and the reports of flaring events became more and more frequent and progressivly more and more complex. No two events were similar, as there were variations observed in source of size, ejections of plasma along with shockwaves driving into the interplanetary space. Advances of radio technology during the second world war ensured detections of prescence of non-thermal electrons in the solar corona, during military radar operations \citep{hey46}. Around the same time S.E. Forbrush noticed ground level cosmic ray enhancement during major solar flares. These discoveries illuded that the flaring events do not only involve the thermal plasma, but is somehow connects with high energy particles and involves the corona. in the 1950s we started observing the Sun in hard X-rays ($\ge~10~keV$) with rockets and baloons. \cite{peterson59} discovered the first hard X-ray emission during a flare in 1958. Later on, it was deduced from the observations of the enhancements observed in radio and hard X-rays that, the the ejected energetic particles may contain a substantial fraction of the intial energy released \citep{brown71}. The hard X-ray is created by the bremsstrahlung of the electrons colliding into dense material, resulting in a power-law energy distribution. The braodband radio emission from 1 to 100 GHz is created from gyrosynchrotron emission. Finally, observations in EUV and soft X-ray($\le$~10~keV) have shown that the energy released form the flare heats the plasma contained in the coronal loops to temperatures beyond 30 MK. 

%% ################################################################# %%
\subsection{Neupert effect}\label{npt_eff}
%% ################################################################# %%

\cite{neupert68} observed a curious corelation that, the soft X-ray flux during the rise phase of the flare is proportional to the time integral of the centimeter radio flux since the start of the flare. The centimeter radio flux is emitted by relativistic electrons. So, later on the same corelation was found between the hard X-ray flux and soft X-ray flux and can be expressed as,

%%%%%%%%%%%%%%%%%%%%%%%%%%%%%%%%
\begin{equation}\label{npt_eff_eq}
    F_{SXR}(t)~\propto~\int_{t_{0}}^{t}~F_{HXR}(t')~dt'
\end{equation}
%%%%%%%%%%%%%%%%%%%%%%%%%%%%%%%%

This emperical relation is known as the ``Neupert effect''. \cite{neupert68} had already suggested that thismay be due to a causal relationship between the thermal plasma and the energetic electrons. The logical explanation is: the soft X-ray mainly originates from a thermal plasma heated by the energy of the flaring event deposited by the flare acclerated electrons. It is important to mention that we know now, eqn.\ref{npt_eff_eq} is only valid if cooling by conduction of radiation is negligible.

%% ################################################################# %%
\subsection{Standard model of solar flare}\label{sol_flr_std_mod}
%% ################################################################# %%

There were several observations like Neupert effect, which would help us to constrain several phenomenon observed in the flares:

%%%%%%%%%%%%%%%%%%%%%%%%%%%%%%%%%%
\begin{figure}[h!]
    \centering
    \includegraphics[width=0.5\linewidth]{Figures/phu_63_8_818_f2.jpg}
    \caption{A schematic diagram of the standard model of solar flare.}
\end{figure}
%%%%%%%%%%%%%%%%%%%%%%%%%%%%%%%%%%

%%%%%%%%%%%%%%%%%
\begin{enumerate}
    \item The magnetic reconnetion happens in corona, releasing the magnetic free energy. Electrons and energetic particles from the reconnetion site are acclerated along the realigned magnetic field lines. The acclerated particles that escape along the open field lines towards the earth gives rise to the particle events seen from earth.
    \item The acclerated particles move along the field lines downwards along the magnetic field lines, giving rise to the microwave observations seen from falres via gyromagnetic radiation.
    \item The acclerated particles hit the Chromosphere which is considerably denser than the Corona, giving rise to Hard X-ray and gamma ray observed from the footpoints. This also explains why the coronal hard X-ray source is considerably softer than the footpoints. The energetic particles deposit their energy into the local Chromosphere, as they go through series of collision and eventually thermalize.
    \item  As the plasma thermalizes, it starts emitting in thermal bremsstrahlung giving rise to the soft X-ray observed. Also, as the energy is deposited into the Chromosphere from the acclerated particles, it heats the local Chromosphere environment it gradually increases the local pressure. 
    \item As the pressure grows, when the pressure gradient builds up enough the local plasma starts expanding upwards(essentially due to buoyancy) and slowly fills up the coronal loops with soft X-ray emitting plasma. This phenomenon is known as "Chromospheric evaporation". This was dorectly observed later on, in blue-shifted lines of hot material.
    \item This whole scenario explains the Neupert effect. As the energy form the acclerated particles is converted into the soft X-ray emitting plasma, and it builds up over time. That explains the soft X-ray being proportional to the time integral of the hard X-ray flux.
\end{enumerate}
%%%%%%%%%%%%%%%%%

This whole scenario is known as the ``Standard model of solar flare''. While the standard model is an attempt to explain and unify various kind of differences seen from the neumerous flares we observe, there are several cases where the stnadard model cannot explain the observations. For example, there have been observations of falres where the hard X-ray footpoint do not form. It is proposed that in these cases the plasma in the coronal loops is so dense that the acclerated particle collide enough to thermalize within the flare loops before reaching the Chromosphere, deffusing the energy more evenly within the loop, rather than dumping it at the base of the loops \citep{veronig02,veronig04}. Another flare which previously occured at the same region might explain the denser coronal loops \citep{strong84,bone07}.  

%% ################################################################# %%
\subsection{The energectics of solar flares}\label{sol_flr_energ}
%% ################################################################# %%

With an experience of observing solar flares for more than 150 years, remarkably enough, we have barely started scratching the surface of the complexity involved with the solar flares. The reconfiguring of magnetic structure, which almost always involves complex geometry, making almost all events unique in some sense. After that the released magnetic free energy is transported accross various layers of the sun and converted into various other forms of energy. Consodering how the energy is transformed the magnetic energy of the active region that is released after the reconnection into the reconnection outflow jets, the kinetic energy of escaping particles, the thermal and the kinetic energy of the Chromospheric plasma evaporating, the radiative and conductive losses. In case of the eruptive events, there is the added complexity of the kinetic and potential energy of the CMEs, the enrgy of the shcoks and the kinetic energy of the solar energetic particles. 

Inorder to constrain the models of solar eruptions and various nuanced aspects of it, like magnetic reconnection, particle accleration, heating etc. detailed quantitative characterization is absolutely necessary. There have been several studies that have tried to quantify the partition between various subsets of the energies. The questions that are perticularly important are:

%%%%%%%%%%%%%%%%%%%%%%%%%%%%%%%%%%%%%%%%%%%%%
\begin{itemize}
    \item \textbf{If an active region can have enough free enrgy to account for the total energy relased in the solar flares and/or CMEs.}
    \item \textbf{What is the energy partition betwen flares and CMEs.}
    \item \textbf{If the non-thermal component have enough energy to power up the thermal component.}
\end{itemize}
%%%%%%%%%%%%%%%%%%%%%%%%%%%%%%%%%%%%%%%%%%%%%

It is generally well known by now that the active region have enough magnetic free energy to power the flare and CME \citep{emslie12,ash17}. The partition of energy between the flare and associated CME is much more fuzzy. \cite{emslie12} found that the flare and CME have energies of same order of magnitude, while \cite{ash17} concluded the flare dominates the CME in-terms of the energy. However the simple question, whether the non-thermal component of the flares have enough enrgy is still not resolved as even the most recent studies contradict each other in the most puzzling fashion. Here I will discuss the contradictions arising from some of the studies\citep{stosire07,emslie12,inglis14,warmuth16a,warmuth16b,ash17}. The details of the studies can be summarized as follows:

%%%%%%%%%%%%%%%%%%%%%%%%%%%%%%%%%%%%%%%%%%%%%%%
\begin{table}[h!]
    \centering
    \hline 
    \hline
    \resizebox{\textwidth}{!}{%
    \begin{tabular}{||c|c|c|c|c|c|c||}
       Study & No. of flares & GOES class range & Thermal model & Thermal spectrum & Thermal volume & Thermal losses \\
       \cite{stosire07} & 18 & A3-B7 & Isotherm. & RHESSI & TRACE & X \\
       \cite{emslie12} & 38 & C5-X28 & Isotherm. & RHESSI & RHESSI & Rad. \\
       \cite{inglis14} & 10 & B3-B9 & Multitherm. & RHESSI+AIA & RHESSI & Rad. \\
       \cite{warmuth16a,warmuth16b} & 24 & C3-X17 & Isotherm. & RHESSI+GOES & RHESSI & Rad.,Cond. \\
       \cite{ash17} & 188 & M1-X7 & Multitherm. & AIA & AIA & Rad. \\
       \hline
    \end{tabular}}
    \caption{The details of the studies.}
    \label{tab1}
\end{table}
%%%%%%%%%%%%%%%%%%%%%%%%%%%%%%%%%%%%%%%%%%%%%%%


%% ################################################################# %%
\section{Motivation}\label{sec:mot}
%% ################################################################# %%

%% ################################################################# %%
\section{Outline of Thesis}\label{sec:outline}
%% ################################################################# %%