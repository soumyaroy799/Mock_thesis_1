%\begin{abstract}

 %   The \ion{Mg}{2}~k \& h line intensity ratios can be used to probe the characteristics of the plasma in the solar atmosphere. In this study, using the observations recorded by the Interface Region Imaging Spectrometer (IRIS), we study the variation of the \ion{Mg}{2}~k \& h intensity ratio for three flares belonging to X-class, M-class, and C-class, throughout their evolution. We also study the k-to-h intensity ratio as a function of magnetic flux density obtained from the line-of-sight magnetograms recorded by the Helioseismic and Magnetic Imager (HMI) on board the Solar Dynamics Observatory (SDO). Our results reveal that while the intensity ratios are independent of magnetic flux density, they show significant changes during the evolution of the C-class and M-class flares. The intensity ratios start to increase at the start of the flare and peak during the impulsive phase before the flare peak and decrease rapidly thereafter. The values of the ratios fall even below the pre-flare level during the peak and decline phases of the flare. These results are important in the light of heating and cooling of localized plasma and provide further constraint on the understanding of flare physics.
    
%\end{abstract}

%%----------------------------------------------------
\section{Introduction} \label{sec:intro}
%%----------------------------------------------------

Solar flares are the most energetic events on the Sun, where an enormous amount of magnetic free energy is released due to the reconfiguration of the coronal magnetic field. The released energy can cause particle acceleration, heating and flows in the solar atmosphere and a transient enhancement in solar radiative output. It is observed that most of the energy radiated in flares originates from the dense chromosphere \citep{fletcher10,milligan14}. Hence, studying the chromospheric lines during flares provides us with diagnostics, which may be important for understanding the physics of solar flares and their effect on the local plasma environment.

The chromosphere emits in various UltraViolet(UV) and optical lines. While many optical lines, e.g., H$\alpha$, \ion{Ca}{2}, are routinely observed from ground-based telescopes, observations in the \ion{Mg}{2} resonance lines have been rare in the past. Since the launch of the Interface Region Imaging Spectrograph \citep[IRIS;][]{IRIS} we have been in the position to monitor these lines regularly with excellent spatial and spectral resolution.

The \ion{Mg}{2}~k and h lines are transitions to the ground state from a finely split pair of upper levels ($3p~^{2}P_{\nicefrac{3}{2}}${--}$3s~^{2}S_{\nicefrac{1}{2}}$ and $3p ^{2}P_{\nicefrac{1}{2}}${--}$3s~^{2}S_{\nicefrac{1}{2}}$). These transitions create the optically thick lines in the wavelengths 2796.34~{\AA} (\ion{Mg}{2}~k) and 2803.52~{\AA} (\ion{Mg}{2}~h). It is suggested that the intensity ratios of these lines can be used to probe the optical depth of the local environment \citep{kerr15}. 

The integrated intensity of a line transitioning from an upper-level j to a lower-level i, is dependent on the collision strength for that transition $\Omega_{ij}$, which is given by~\citep{henri62,mariska92},

%%---------------------------------------------------------------
\begin{equation*}
    \Omega_{ij}~=~\frac{8\pi}{\sqrt{3}}~\frac{I_{H}}{\Delta \epsilon_{ij}}~g~\omega_{i}~f_{ij}
\end{equation*}
%%---------------------------------------------------------------

\noindent Where $I_{H}$ is the ionization energy of Hydrogen, $\Delta \epsilon_{ij}$ is the threshold energy for the transition, g is the Gaunt factor, $\omega_{i}$ is the statistical weight of the level and $f_{ij}$ is the oscillator strength. In optically thin conditions, the intensity ratio of the k to h line is the ratio of the collision strengths, as the escape probability of photon is unity. The \ion{Mg}{2} k and h lines share the same ionization state and originate from a transition to a shared lowered level. As the statistical weight ($\omega_{i}$) is same in both cases, the line intensity ratio is simply the ratio of the oscillator strengths($f_{ij}$). This implies the ratio is 2:1 in optically thin conditions, and lower when the medium is optically thick \citep{kerr15,levens19}.

In addition, the \ion{Mg}{2} k and h lines can be used to estimate the velocity in the middle and upper chromosphere, the chromospheric velocity gradients, the temperature in the middle chromosphere \citep{leenarts13a,leenarts13b,pereira13}. The \ion{Mg}{2} triplets in emission can be used to identify heating in the lower chromosphere \citep{pereira15}. There have also been multiple studies that have shown a spatial variation of the \ion{Mg}{2} line profiles \citep{dalda23,panos18}. \cite{polito23} showed that the leading edge of the flare ribbon is associated with enhanced broadening and strong central reversal. They interpreted the difference in the profile as a difference in the heating mechanism at the leading edge and bright part of the flare ribbons. \cite{panos21,panos21_2} showed similar differences between the line profiles and energy input.

Using the observations recorded by the OSO-8 LPSP instrument, \citep{lemaire84} studied the evolution of the intensity ratio of \ion{Mg}{2}~h~\&~k, \ion{Ca}{2}~h~\&~k and Ly~$\alpha$~\&~$\beta$ lines. \cite{lemaire84} showed that the intensity ratio of the \ion{Ca}{2}~k/h lines increased from 1 to 1.2 during the ascending phase of a flare and returned to 1 during the later phases. The authors interpreted the correlated temporal behavior across various elements as an indication of downward energy propagation. We note that this may suggest a slight decrease in the opacity due to localized heating at the formation height of the \ion{Ca}{2} line during the rise phase of the flare.

%\sout{In spite of the excellent potential of these spectral lines in diagnosing the localized atmosphere, such} \sout{studies have not been taken up in detail due to the scarcity of spectral observations in these lines.}

Here, we study the evolution of the intensities ratios of the \ion{Mg}{2}~h~\&~k lines during the course of the evolution of three flares, \textit{viz.}, C-class, M-Class and X-class. We focus on the the dependence of line ratios on the underlying magnetic field strength, which to our knowledge has not been explored so far. The rest of the paper is structured as follows. \S\ref{sec:obs} discusses the observations used in this paper. We discuss how we reduce and analyze the data and the results in \S\ref{sec:dar}.