\noindent
The journey of a PhD is like a marathon. There are times when you feel like sprinting down the lane, full of energy, driven by small successes, getting stuck on something for a while, and finally finding that "eureka" moment. Other times, you feel out of breath, barely dragging your legs along, just trying to stand. I believe what matters most are the connections you make with people who push you and help you during those difficult moments—because without them, I wouldn't be here writing this acknowledgment!  

\noindent
First, I would like to thank Durgesh; this thesis would not have been possible without his guidance. He gave me enough freedom to explore whatever I wanted, offering valuable suggestions along the way but never giving "orders." More importantly, he carefully considered any differing viewpoints I raised from the first day. Without this freedom, I probably wouldn't be as proud of my growth over the past five years. Very few people get the opportunity to contribute to one of their country's first major observatory-class missions. I am especially grateful for the confidence he and Ram placed not only in me but also in several other young students while realizing SUIT. While I worked on the project for only five years, I also want to thank everyone who was part of it before me—without their hard work, I would not have been able to take over so easily.  

\noindent
Next, I would like to thank Kathy. Just like Durgesh, she also allowed me the freedom to explore the problems I was given, nudging me in the right direction whenever I got stuck. I also want to thank Chris. His enthusiasm in answering my questions and our discussions on various topics were invaluable. Working at CfA was a remarkable experience. I am grateful to Kathy, Durgesh, the Smithsonian Institution, and CfA for making this possible. Experiencing the life of a graduate student in two very different systems gave me insight into both the challenges and advantages of each. Beyond being a scientific opportunity, it was a profound lesson in becoming an independent person not just a "scientist". Leaving your country and loved ones after twenty-six years to live alone, work, cook, and function as an independent adult in a foreign country is not easy. But those one-and-a-half years helped me mature quickly.  

\noindent
Our memories of a place are deeply tied to the people and the experiences we share with them. I want to thank Tatiana and Chad specifically; I will forever cherish the fun we had in the chess club. Those library gatherings early on at CfA were something I looked forward to when I had not yet made many friends. Tatiana, you were a godsend! As someone who isn't very extroverted, you made it easy for me to open up to people at CfA. I also want to thank several people involved in the HSO-Connect project, with whom I had many fruitful academic and non-academic discussions: Sophie, Dana, Ritesh, Xiaoyan, Crisel, Dan, Sam, and Yeimy. Similarly, my time at IUCAA became equally memorable, thanks to many people—whether we were traveling, hiking, playing board games and football, cooking together, or organizing Diwali celebrations for two years in a row.  

\noindent
In the early days, I spent a lot of time with Sunil and Sorabh playing FIFA in my room or on the pantry TV. It was a relief to know I wasn’t the only one "wasting" time on "silly video games"! Tathagata, Suprovo, Sorabh, Sunil, Somak Da, Sourabh, Suraj, Dhruv, we spent so much time playing board games I had not even known existed. Ankush, Partha, Biku, Kavita, and Meenakshi, our cooking sessions and hangouts were respites I looked forward to. A special shout out to the football gang {—-} Sunil, Sorabh, Somak Da, Sayak, and Soumil. Thanks to you all, we managed to follow football regularly, despite our busy schedules.  

\noindent
I also thank the solar group and its wonderful members for both academic and non-academic discussions over the years: Nived, Vishal, Janmejoy, Megha, Sneha, Rushikesh, Rahul, Sreejith, Abhishek, Deepak, Sargam, and Avyarthana. Megha and Anirban felt like older siblings I could always approach when needed.  

\noindent
A special shout-out to Nived! Although we didn't spend much time together, I had more scientific discussions with you than anyone else. I respected your opinions so much that I started running every crazy idea by you first. Your constant presence in my scientific journey has been invaluable.  

\noindent
Special thanks to Lindsay Glesener. Many people never realize the impact they can have on others. During the 2023 SF AGU meeting, Chris took me with his group to Berkeley to see the FOXSI-4 assembled. At the time, I was at a low point, doubting my ability to succeed in academia and considering leaving after my PhD. Lindsay shared how, as a PhD student, she accidentally destroyed the CCD hours before the FOXSI-1 launch, delaying it. Despite that, she went on to become the PI of FOXSI-4. Her story rekindled my confidence. Lindsay, I may never tell you this in person, but however long I stay in academia, your story will remain a key source of inspiration.  

\noindent
I am deeply grateful to my professors at Presidency University, Kolkata {--} Suchetana and Ritaban for their support during my bachelor’s and master’s programs. Without them, I might not have pursued a PhD in astronomy.  

\noindent
This work would not have been possible without the support of my family and friends. I owe my career in physics to the enthusiasm and encouragement of my uncle, Maloy Kumar Roy, and my friend’s father, Ranjan Basak. My uncle Shyamal Kumar Ray has been a constant companion, whether watching football or movies together. Aloke Mitra has been a lifelong pillar of support. Sayan Chakraborty, who started as my teacher, gave me the coding skills that became invaluable during my PhD.  

\noindent
Finally, I want to thank my parents. They gave me the freedom to explore whatever I wanted from a very early age. After high school, they supported my decision to pursue a career in basic science, even when friends and relatives were skeptical.  

\noindent
This thesis would not have been possible without the open data policies of several instruments. I acknowledge the use of data from AIA, HMI, IRIS, XRT, STIX, Chandrayaan-2/XSM, Stereo-A, GOES/XRS, and GOES/SUVI. I also thank the SUIT and Pegasus servers at IUCAA for providing computational facilities.
